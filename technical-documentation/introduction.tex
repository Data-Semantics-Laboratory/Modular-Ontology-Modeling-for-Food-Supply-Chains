\chapter{Overview}

We are presenting core ontology modules aimed at food supply chain tracking related to grains. This work was performed under financial assistance award 70NANB19H094 from U.S. Department of Commerce, National Institute of Standards and Technology.

Development of the modules was a collaborative effort and was carried out using the principles laid out in, e.g., \cite{KrisnadhiH16,KrisnadhiKHARJ16,enslaved-jws,iccs18-invited}. The modeling team included domain experts, data experts, software developers, and ontology engineers. 

The (partial) ontology constituted by these modules has, in particular, been developed as a \emph{modular} ontology \cite{HitzlerGJKP17,iccs18-invited} based on ontology design patterns \cite{HGJKP2016,MODL}. This means, in a nutshell, that we first identified key terms relating to the data content and expert perspectives on the domain to be modeled, and then developed ontology modules for these terms. The resulting modules, which were informed by corresponding ontology design patterns, are listed and discussed in Chapter \ref{sec:mods}. In Chapter \ref{chap:openissues} we discuss possible next steps.

A use case description which drove the development of these modules can be phrased as follows: \emph{The end goal is to provide queryable (meta)data which makes it possible to trace agri-food supply chains relevant to grains. For this, we will develop an ontology which is an adequate data schema for all relevant data, from all relevant sources, that is needed to trace these supply chains. Of relevance is all data that pertains to (possible) food contamination that may affect health.}

A more restricted perspective on the use case -- which is still adequate for the modules provided herein -- is: \emph{Identify all Traceable Resource Units (TRUs), relevant events, containers, which are involved in the past (or future) of a given TRU, together with the information how these TRUs, events, containers relate to the given TRU.}

For background regarding Semantic Web standards, in particular the Web Ontology Language OWL, including its relation to description logics, we refer the reader to \cite{owl2-primer, FOST}.

All feedback is most welcome, and can be directed to Cogan Shimizu, coganmshimizu@ksu.edu, and Pascal Hitzler, hitzler@ksu.edu.


